\documentclass{article}
\usepackage[margin=18.5mm, a4paper]{geometry}
\usepackage{xcolor}
\usepackage{tcolorbox}
\usepackage{listings}
\usepackage{graphicx}
\usepackage{multicol}
\title{Bibliographic Study on Multi-Agent Path-Planning}
\author{Erwin Lejeune, Sampreet Sarkar \\ \small{M2 Control and Robotics: Embedded Real-Time Systems,}\\\small{\textit{\'Ecole Centrale de Nantes}}}
\date{}


\begin{document}
\maketitle
\nocite{*}

\section{Introduction}
Multi-agent path planning \textit{(MAPP)} has been widely considered to be a PSPACE-hard problem. The general description of the problem involves the proposal of paths\textit{(optimal or sub-optimal)} for each unit\textit{(mobile robots, UAVs, game characters, etc.)} from a starting position to a goal position in 2-D or 3-D space, whilst avoiding obstacles, which might be static or dynamic. The applications of MAPP are very diverse, and perhaps one of the best examples that we can consider is that of the air traffic control at any airport. In the process of trying to automate the planning, we observe that standard algorithms like A* and RRT, which are the go-to solutions for single-agent path planning, do not perform very well. This is partially because of the fact that these algorithms do not possess a great sense of scalability. In such cases, a global search is often impractical, even if there are only a small number of units in consideration. With an increase in the research and development of swarm-robotic applications, we have to also consider collective actions for a fleet of units and convey the appropriate messages to generate collision-free paths for each individual unit, and the fleet in turn will collectively accomplish the task.\\

One of the main challenges encountered during the solution og MAPP problems is to be able to generate paths that are free of conflicts. In the context of UAVs, the solution turns out to be effective designing of Conflict Detection and Resolution\textit{(CDR)} algorithms. In such a scenario, the algorithm has to generate multiple UAV paths, while maintaining a minimum distance of separation required for safe operation. In a more general context, conflict detection and resolution may be done online or offline, depending upon the agents and the environment they are interacting in. However, through the course of this document, we will see that sometimes a mix of online and offline approach is the most favourable.\\

The construction of the optimal strategy for the agent would require some knowledge about the problem | a description of the agent, and how it interacts with the environment, a description of the domain or the environment the agent will function in, and a problem statement, which presents the current state of the agent, and the final state the agent wants to be in. However, in further pondering of the problem statement, we discover that this in not enough information. We observe that the current states of the other agents in the environment is also a necessary piece of information. One can think of some approaches we might take to solve this problem, namely manual abstraction of the search space, and decomposition of the bigger search problems into multiple smaller search problems. These approaches are simple, but they eventually run into deadlocks. The general trend to these types of problems have been to somehow balance the completeness and optimality factors, while still retaining enough performance to be practical, which will be discussed later.\\

A standard description of the search space with respect to low altitude airspace domain has been provided in \cite{ho2019multi}, and it is interesting, how the authors have managed to preserve the heterogenity of the units\textit{(in their case, UAVs)}, and provide solvers for Multi-agent Path Finding. Similarly, the authors in \cite{bhattacharya2010multi} have provided a similar algorithm for the domain of mobile robots, where they dealt with the collision avoidance and path planning problems by enforcing a minimum distance of separation between pairs of robots. The assumptions that the authors have worked under were that the robots have been independently assigned a set of tasks in an unordered fashion. This becomes important when we consider the scenario with respect to swarm robotics, where a group of robots collectively accomplish a task or a set of tasks. If the robots are to operate as a swarm, then care must be taken to ensure that we have a system of acknowledgement that a certain task has indeed been completed by a robot in the swarm. This approach, while raising complexity of the general algorithm, ensures that we have a certain synchronisation in the operation.\\

The outline of the report might be summarized as follows | In the upcoming section, we will establish the problem statement further, with mathematical models and supporting equations. We will survey the key algorithms and highlight the positive and negative aspects of each approach in the Key Results and Arguments section, and we will present our findings on the topic and propose modifications to existing algorithms in the Conclusion.
\section{Problem Statement}

\section{Key Results and Arguments}

\section{Conclusion}


\bibliographystyle{unsrt}
\bibliography{references}
\end{document}
